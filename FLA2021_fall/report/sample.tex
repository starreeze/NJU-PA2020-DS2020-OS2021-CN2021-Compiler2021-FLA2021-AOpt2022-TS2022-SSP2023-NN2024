\documentclass[11pt,a4paper,UTF8]{article}
\usepackage{ctex} % Chinese support
\usepackage{graphicx} % Insert images
\usepackage{listings} % Print source code
\usepackage{color} % Color support
\usepackage{booktabs} % Professional table support
\usepackage{pdflscape} % Landscape pages support in PDF
\usepackage{hyperref} % Hypertext links support for cross-referencing

% Customize hyperref format (it's set to no special format here)
\hypersetup{hidelinks}

% Declare directories to search for graphics files for graphicx
\graphicspath{{figures/}{logo/}}

% Define source code style for listings
\lstdefinestyle{c}{
  language=C,
  basicstyle=\ttfamily\footnotesize,
  keywordstyle=\bfseries\color[rgb]{0, 0, 1},
  identifierstyle=\color[rgb]{0, 0, 0.45},
  stringstyle=\color[rgb]{0.6, 0.1, 0.1},
  commentstyle=\itshape\color[rgb]{0.05, 0.5, 0.05},
  backgroundcolor=\color[gray]{0.95},
  numbers=left,
  numbersep=5pt,
  numberstyle=\color[gray]{0.6},
  breaklines=true
}
\lstdefinestyle{cpp}{
	language=C++,
	basicstyle=\ttfamily\footnotesize,
	keywordstyle=\bfseries\color[rgb]{0, 0, 1},
	identifierstyle=\color[rgb]{0, 0, 0.45},
	stringstyle=\color[rgb]{0.6, 0.1, 0.1},
	commentstyle=\itshape\color[rgb]{0.05, 0.5, 0.05},
	backgroundcolor=\color[gray]{0.95},
	numbers=left,
	numbersep=5pt,
	numberstyle=\color[gray]{0.6},
	breaklines=true
}

% Define new command for title page
\newcommand{\reporttitle}[2]{
  \LARGE\textsf{#1}\quad\underline{\makebox[12em]{#2}}
}
\newcommand{\reportinfo}[2]{
  \large\makebox[4em]{\textsf{#1}}\quad\underline{\makebox[18em]{#2}}
}

% The document begins here
\begin{document}
\begin{titlepage}
  \centering
  \vspace*{\fill}
  \includegraphics[height=144pt]{nju-logo}\\[48pt]
  {\huge\textsf{课\ 程\ 实\ 验\ 报\ 告}}\\[48pt]
  \reporttitle{实验名称}{图灵机模拟和编程}\\[72pt]

  \reportinfo{课程名称}{形式语言与自动机}\\[8pt]
  \reportinfo{院\hspace{\fill}系}{计算机科学与技术系}\\[8pt]
  \reportinfo{学\hspace{\fill}号}{191220129}\\[8pt]
  \reportinfo{姓\hspace{\fill}名}{邢尚禹}\\[8pt]
  \reportinfo{邮\hspace{\fill}箱}{191220129@smail.nju.edu.cn}\\[8pt]
  \reportinfo{实验日期}{2021年12月}\\
  \vspace*{\fill}
\end{titlepage}

\tableofcontents
\newpage

\section{基本信息}
\subsection{实验进度}
本次实验完成了所有内容,包括选做的verbose模式。具体地,
\begin{itemize}
	\item 图灵机的解析和模拟运行;
	\item 详尽的报错信息,程序健壮性较高;
	\item 公约数的图灵编程。
\end{itemize}

\subsection{编译要求}
\par 项目提供Makefile,可直接make,在ubuntu20.04(kernel 5.11.0-34), g++9.3.0环境下可编译并正确运行示例代码和gcd.tm。

\section{实验内容}
\subsection{图灵机的解析和模拟}
图灵机本身的状态转移并不复杂,关键在于能够正确解析,遇到错误时提供详细的报错信息,不能崩溃。
\subsubsection{数据结构}
为便于报错,首先使用一个全局变量error来记录当前的文件名、行号、读取的内容等。
\lstinputlisting[style=cpp]{1.cpp}
\par 而图灵机的结构则相对复杂,其数据和使用的数据结构如下。
\begin{itemize}
	\item 状态集:集合;
	\item 终止状态集:集合;
	\item 输入字符集:集合;
	\item 纸带字符集:集合;
	\item 初始状态:字符串;
	\item n条纸带:数组,元素为一条纸带,每条纸带是字符的链表(选择链表是为了便于首尾插入);
	\item n个读头:数组,元素为对应纸带的迭代器;
	\item 转移函数:自定义数据类型,要实现对给定的(state, input)查找(new\_state, new\_symbol, direction)。
\end{itemize}
\lstinputlisting[style=cpp]{2.cpp}
\par 对于转移函数,关键是要存储从(state, input)到(new\_state, new\_symbol, direction)的对应关系。我设计了下标对应的方式,用一维数组存储state,二维数组存储input,其第一维度与state下标对应,二维数组存储action(即上述三元组),其第一维度与state下标对应,第二维度与input对应。
\lstinputlisting[style=cpp]{3.cpp}
\par 其实本来我想用unordered\_map来实现,但后来发现还要做通配符匹配,涉及优先级(越少通配符的匹配越优先)等,不便于使用unordered\_map来实现。于是我采用了数组存储,每次查找相应匹配时就直接遍历,找到通配符最少的匹配的返回。
\lstinputlisting[style=cpp]{4.cpp}

\subsubsection{解析器}
解析器的重点在于字符串处理,其实这一方面C++不是特别擅长。例如,要实现把字符串按照空格分给进一个数组的功能,Python只需要一句res = str.split(' '),但C++要这么多:
\lstinputlisting[style=cpp]{5.cpp}
所以还是希望以后能允许自选实现语言。
\par 解析器主要是要考虑到各种异常情况,比较复杂,但实现起来没有任何难度,具体过程不再赘述。

\subsubsection{模拟器}
在实现了上述数据结构之后,模拟器也非常简单,以至于大部分代码都在处理verbose模式的打印,真正的状态转换就几十行。事实上,直接按照上述的匹配规则调用匹配函数即可。
\lstinputlisting[style=cpp]{6.cpp}

\subsection{GCD图灵编程}
实现GCD比回文串检测复杂得多,使用辗转相减法,需要大量的移动和作差操作。我是使用的2-tape,实现的流程如下:
\begin{enumerate}
	\item 将0右侧的串移动到tape-2,并将'0'替换为blank;
	\item 两个读头都移动到最左侧,在两个tape的最左侧(开始位置左一格)写上'0';
	\item 作差,即将公共长度部分的'1'全部替换为blank;
	\item 如果长度相等,转8;否则在较短串的结束位置写上'0',将其读头移动到最左侧的'0'的右边一格;
	\item 将较长的串的剩余的'1'移动到短串上,如果要覆盖右侧的'0',则将其改为'2';
	\item 将其读头移动到右侧的'0'或'2'上,向左移动到左侧的'0'上,将纸带上的字符替换成'1';于此同时,上述每一步操作时也将长串的读头右移一格,并写下'1',使得最终'1'的长度与短串上左侧'0'与右侧'0'/'2'的距离相等;
	\item 将长串的读头移动到最左侧的'0'的右边一格,转3;
	\item 将tape-1的读头向左移动到左侧'0'的位置,并写下'1',删除两侧的'0'。
\end{enumerate}
具体的代码见programs/gcd.tm。

\section{总结与建议}
\begin{itemize}
	\item C++在处理字符串时不太好用,因此强烈建议允许使用其他语言或第三方库;
	\item 个人认为实现各种错误处理以增强健壮性意义不大,并不能非常好地训练学生设计思想和编程能力,只能给学生增加额外负担(至少我的体验在写异常处理时是非常不好的);
	\item 这次实验的总体工作量还比较大,如果能够早一点布置,我们的时间安排会更舒适,否则快要期末了可能有部分同学来不及完成。
\end{itemize}

\end{document}