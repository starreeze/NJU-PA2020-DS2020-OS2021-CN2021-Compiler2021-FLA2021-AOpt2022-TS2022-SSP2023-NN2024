\documentclass[a4paper,UTF8]{article}
\usepackage{ctex}
\usepackage[margin=1.25in]{geometry}
\usepackage{color}
\usepackage{graphicx}
\usepackage{amssymb}
\usepackage{amsmath}
\usepackage{amsthm}
\usepackage{enumerate}
\usepackage{bm}
\usepackage{hyperref}
\usepackage{epsfig}
\usepackage{color}
\usepackage{booktabs}
\usepackage{tcolorbox}
\usepackage{mdframed}
\usepackage{lipsum}
\newmdtheoremenv{thm-box}{myThm}
\newmdtheoremenv{prop-box}{Proposition}
\newmdtheoremenv{def-box}{定义}

\setlength{\evensidemargin}{.25in}
\setlength{\textwidth}{6in}
\setlength{\topmargin}{-0.5in}
\setlength{\topmargin}{-0.5in}
% \setlength{\textheight}{9.5in}
%%%%%%%%%%%%%%%%%%此处用于设置页眉页脚%%%%%%%%%%%%%%%%%%
\usepackage{fancyhdr}
\usepackage{lastpage}
\usepackage{layout}
\footskip = 10pt
\pagestyle{fancy}                    % 设置页眉
\lhead{2022年秋季}
\chead{时间序列分析}
% \rhead{第\thepage/\pageref{LastPage}页}
\rhead{作业二}
\cfoot{\thepage}
\renewcommand{\headrulewidth}{1pt}  			%页眉线宽,设为0可以去页眉线
\setlength{\skip\footins}{0.5cm}    			%脚注与正文的距离
\renewcommand{\footrulewidth}{0pt}  			%页脚线宽,设为0可以去页脚线

\makeatletter 									%设置双线页眉
\def\headrule{{\if@fancyplain\let\headrulewidth\plainheadrulewidth\fi%
\hrule\@height 1.0pt \@width\headwidth\vskip1pt	%上面线为1pt粗
\hrule\@height 0.5pt\@width\headwidth  			%下面0.5pt粗
\vskip-2\headrulewidth\vskip-1pt}      			%两条线的距离1pt
 \vspace{6mm}}     								%双线与下面正文之间的垂直间距
\makeatother

%%%%%%%%%%%%%%%%%%%%%%%%%%%%%%%%%%%%%%%%%%%%%%
\numberwithin{equation}{section}
%\usepackage[thmmarks, amsmath, thref]{ntheorem}
\newtheorem{myThm}{myThm}
\newtheorem*{myDef}{Definition}
\newtheorem*{mySol}{Solution}
\newtheorem*{myProof}{Proof}
\newcommand{\indep}{\rotatebox[origin=c]{90}{$\models$}}
\newcommand*\diff{\mathop{}\!\mathrm{d}}

\usepackage{multirow}

%--

%--
\begin{document}
\title{时间序列分析\\
作业二}
\author{191220129, 邢尚禹, starreeze@foxmail.com}
\maketitle

\section*{作业提交注意事项}
\begin{tcolorbox}
\begin{enumerate}
  \item[(1)] 请严格参照教学立方网站所述提交作业,文件命名统一为{\color{red}学号\_姓名.pdf};
  \item[(2)] 未按照要求提交作业,或提交作业格式不正确,将会被扣除部分作业分数;
  \item[(3)] 除非有特殊情况(如因病缓交),否则截止时间后不接收作业,本次作业记零分。
\end{enumerate}
\end{tcolorbox}

\begin{enumerate}
\item 考虑线性模型 $Y=\alpha+\beta X+\varepsilon$ ,其中 $\alpha$ 和 $\beta$ 都是常量。 $X$ 和 $\varepsilon$ 是互 不相关的随机变量, 均值和方差如下: $\mathrm{E}[X]=\mu_{X}, E[\varepsilon]=0$, $\operatorname{Var}[X]=\sigma_{X}^{2}, \operatorname{Var}[\varepsilon]=\sigma_{\varepsilon}^{2}$ 。求:
	\begin{enumerate}[1)]
		\item $\mathrm{E}[Y \mid X]$ 和 $\operatorname{Var}[Y \mid X]$
		\item $\mathrm{E}[Y]$ 和 $\operatorname{Var}[Y]$
	\end{enumerate}

\begin{mySol}
~\\
    \begin{enumerate}[1)]
	\item \begin{align*}
        \mathrm{E}[Y \mid X] &= \mathrm{E}[\alpha+\beta X+\varepsilon \mid X] \\
        &= \alpha+\beta X+\mathrm{E}[\varepsilon \mid X] \\
        &= \alpha+\beta X
    \end{align*}
    \begin{align*}
        \operatorname{Var}[Y \mid X] &= \operatorname{Var}[\alpha+\beta X+\varepsilon \mid X] \\
        &= \operatorname{Var}[\varepsilon \mid X] \\
        &= \sigma_{\varepsilon}^{2}
    \end{align*}
    \item \begin{align*}
        \mathrm{E}[Y] &= \mathrm{E}[\alpha+\beta X+\varepsilon] \\
        &= \alpha+\beta \mathrm{E}[X]+\mathrm{E}[\varepsilon \mid X] \\
        &= \alpha+\beta \mu_{X}
    \end{align*}
    \begin{align*}
        \operatorname{Var}[Y] &= \operatorname{Var}[\alpha+\beta X+\varepsilon] \\
        &= \beta^2 \sigma_{X}^{2} + \sigma_{\varepsilon}^{2}
    \end{align*}
    \end{enumerate}


\end{mySol}



\item 	\begin{enumerate}[1)]
		\item 已知随机过程

		$$
		X_{t}=\varepsilon_{t}+c\left(\varepsilon_{t-1}+\varepsilon_{t-2}+\cdots\right)
		$$

		其中 $\mathrm{c}$ 是一个常量, $\left\{\varepsilon_{t}\right\}$ 是白噪声, 证明这一随机过程是非平稳的。
		\item 基于 $X_{t}$ , 引入新的过程
		$$
		Y_{t}=\nabla X_{t}
		$$

		请证明 $\left\{Y_{t}\right\}$ 是一个 MA(1) 过程, 并说明 $\left\{Y_{t}\right\}$ 是否是平稳的。
		\item 求 $\left\{Y_{t}\right\}$ 的 $A C F$ 。
	\end{enumerate}


 \begin{mySol}
 ~\\
 	\begin{enumerate}[1)]
         \item
             \begin{align*}
                  \operatorname{Var}\left[Y_t\right] &=\operatorname{Var}\left[\varepsilon_t+c\left(\varepsilon_{t-1}+\varepsilon_{t-2}+\cdots\right)\right] \\
                  &=\operatorname{Var}\left[\varepsilon_t\right]+\operatorname{Var}\left[c\left(\varepsilon_{t-1}+\varepsilon_{t-2}+\cdots\right)\right] \\
                  &=\operatorname{Var}\left[\varepsilon_t\right] + c^2\sum_{i=0}^{t-1} \operatorname{Var}\left[\varepsilon_i\right]
             \end{align*}
         由于二阶矩和时间有关,所以是非平稳的。
         \item \begin{align*}
             Y_t&=X_t-X_{t-1} \\
             &= \varepsilon_{t} + (1-c)\varepsilon_{t-1}
         \end{align*}
         因此是$MA(1)$过程.
         $$
             \psi(z)=\sum_{i=0}^{\infty}\psi_iz^{-i}
             = 1+\frac{1-c}{z}
         $$
         由于当$|z|>1$时收敛,所以是平稳的。
         \item \begin{align*}
             \rho_{YY}(k)&=\frac{\gamma_{YY}(k)}{\sigma_Y^2}=\frac{E(Y_tY_{t-k})-E(Y_t)E(Y_{t-k})}{\operatorname{Var}[\varepsilon_t+(1-c)\varepsilon_{t-1}]} \\
             &= \frac{E([\varepsilon_t+(1-c)\varepsilon_{t-1}][\varepsilon_{t-k}+(1-c)\varepsilon_{t-k-1}])}{[1+(1-c)^2]\sigma_{\varepsilon}} \\
             \rho_{YY}(1) &= \frac{E([\varepsilon_t+(1-c)\varepsilon_{t-1}][\varepsilon_{t-1}+(1-c)\varepsilon_{t-2}])}{[1+(1-c)^2]\sigma_{\varepsilon}} \\
             &= \frac{(1-c)E(\varepsilon^2)}{[1+(1-c)^2]\sigma_{\varepsilon}} = \frac{1-c}{1+(1-c)^2}
         \end{align*}
         因此,
         \begin{align*}
             \rho_{YY}(k) &= 0, k>1 \\
             \rho_{YY}(1) &= \frac{1-c}{1+(1-c)^2} \\
             \rho_{YY}(0) &= 1
         \end{align*}
     \end{enumerate}

 \end{mySol}



\item 常数均值模型, $Y_{t}=\mu+\varepsilon_{t} \quad (t=1, \ldots, N) \quad \varepsilon_{t}$ 为 i.i.d., 均值为 0 , 常数方差 $\sigma^{2}$ (白噪声)。对于所有样本,

$$
\hat{\mu}=\frac{1}{N} \sum_{t=1}^{N} Y_{t}=\bar{y}
$$

预测 $\hat{Y}_{N+\ell \mid N}=\hat{\mu}_{\text {。 }}$ 请证明: 预测误差的方差为

$$
V\left[Y_{N+\ell}-\hat{Y}_{N+\ell \mid N}\right]=\sigma^{2}\left(1+\frac{1}{N}\right)
$$


\begin{mySol}

	\begin{align*}
        V\left[Y_{N+\ell}-\hat{Y}_{N+\ell \mid N}\right] &= V[\mu-\hat{\mu}+\varepsilon_{N+l}] \\
        &= V\left[\mu-\frac{1}{N}\sum_{t=1}^{N}(\mu+\varepsilon_{t})+\varepsilon_{N+l}\right] \\
        &= V\left[\varepsilon_{N+l}-\frac{1}{N}\sum_{t=1}^{N}(\varepsilon_{t})\right] \\
        &= \sigma^{2}\left(1+\frac{1}{N}\right)
    \end{align*}

\end{mySol}




\item 证明平稳随机过程 $\{Y(t)\}$ 自协方差函数 $\gamma(k)$ 的性质
\begin{enumerate}[1)]
	\item【对称性】 $\gamma(k)=\gamma(-k)$

	\item【规范性】 $|\gamma(k)| \leq \gamma(0)$

	\item【非负定性】 $\forall y, n, t$ , 二次型 $\sum_{i=1}^{n} \sum_{j=1}^{n} y_{i} y_{j} \gamma\left(t_{i}-t_{j}\right) \geq 0$
\end{enumerate}


\begin{mySol}
~\\
    \begin{enumerate}[1)]
    \item $\gamma(k)=E[Y_tY_{t-k}]=E[Y{t-k}Y{t-k+k}]=\gamma(-k)$

    \item 因为 $\mathrm{V}[\lambda_1 Y(t) + \lambda_2 Y(t+k)] \geq 0$,故$(\lambda_1^2 + \lambda_2^2) \gamma(0) + 2\lambda_1 \lambda_2 \gamma(k) \geq 0$ \\
    若$\lambda_1 = \lambda_2 = 1$, 有$\gamma(k) \geq -\gamma(0)$;
    若$\lambda_1 =1, \lambda_2 = -1$,有$\gamma(k) \leq \gamma(0)$;
    因此$|\gamma(k)| \leq \gamma(0)$

    \item
\end{enumerate}

\end{mySol}




\item 开放题:对课程的建议


\begin{mySol}
	作业整体来说有点多,压力太大,希望能减少一点作业量


\end{mySol}

\end{enumerate}

\end{document}